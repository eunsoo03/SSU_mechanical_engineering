\documentclass[a4paper]{scrartcl}
\usepackage[english]{babel}
\usepackage[top=2cm,bottom=3cm,left=2.5cm,right=2.5cm]{geometry}
\usepackage[colorlinks=true, allcolors=black]{hyperref}
\usepackage{wrapfig} %문단 내 이미지 삽입
\usepackage{graphicx} %색상
\usepackage{overpic}
\usepackage[normalem]{ulem}%취소선
\usepackage{array} %표
\usepackage{mdframed, tcolorbox} %글상자
\usepackage[yyyymmdd]{datetime}
	\renewcommand{\dateseparator}{--}
\usepackage{amsmath, amsfonts, amssymb, bm} %수식
	\DeclareMathOperator{\arccsc}{arccsc}
	\DeclareMathOperator{\arcsec}{arcsec}
	\DeclareMathOperator{\arccot}{arccot}
	\DeclareMathOperator{\csch}{csch}
	\DeclareMathOperator{\sech}{sech}
	\DeclareMathOperator{\arcsinh}{arcsinh}
	\DeclareMathOperator{\arccosh}{arccosh}
	\DeclareMathOperator{\arctanh}{arctanh}
	\DeclareMathOperator{\arccsch}{arccsch}
	\DeclareMathOperator{\arcsech}{arcsech}
	\DeclareMathOperator{\arccoth}{arccoth}
	
	\DeclareMathOperator{\meter}{m}
	\DeclareMathOperator{\cm}{cm}
	\DeclareMathOperator{\mm}{mm}
	\DeclareMathOperator{\mum}{\mu m}
	\DeclareMathOperator{\newton}{N}
	\DeclareMathOperator{\kn}{kN}
	\DeclareMathOperator{\kgf}{kgf}
	\DeclareMathOperator{\pa}{Pa}
	\DeclareMathOperator{\kpa}{kPa}
	\DeclareMathOperator{\mpa}{MPa}
	\DeclareMathOperator{\gpa}{GPa}
	\DeclareMathOperator{\knpm}{kN/m}
	\DeclareMathOperator{\kph}{km/h}
	\DeclareMathOperator{\mps}{m/s}
	\DeclareMathOperator{\tkph}{kph}
	\DeclareMathOperator{\tmps}{mps}
	\DeclareMathOperator{\mpss}{m/s^2}
	\DeclareMathOperator{\dgr}{\!^\circ}
	\DeclareMathOperator{\cel}{\!^\circ C}
	\DeclareMathOperator{\kg}{kg}
	\DeclareMathOperator{\kgpcm}{kg/m^3}
	\DeclareMathOperator{\nm}{N\cdot m}
	\DeclareMathOperator{\knm}{kN\cdot m}
	\DeclareMathOperator{\kw}{kW}
	\DeclareMathOperator{\kwh}{kWh}
	\DeclareMathOperator{\mmhg}{mmHg}
	\DeclareMathOperator{\snd}{s}
\usepackage{polynom} %나눗셈 필산
\usepackage{cancel} %수식 약분선
\usepackage{titlesec} %섹션 이름 변경
	\titlespacing*{\section}{3mm}{0mm}{1mm}
	\titleformat{\section}{\bfseries\large}{}{0ex}{}
\usepackage{kotex} %한글

\newcommand{\prob}[2]{\section{#1}\begin{mdframed}#2\end{mdframed}}

\newlength{\picwidth}
\newcommand{\probpic}[4]{
	\setlength{\picwidth}{145mm}\addtolength{\picwidth}{-#3}\section{#1}\begin{mdframed}\begin{tabular}{m{#3}m{\picwidth}}
	\includegraphics[width = #3]{#2} & #4\end{tabular}\end{mdframed}
	}
	
\newcommand{\asw}[2]{
	\begin{flushright}
		#1\quad$\blacktriangleleft$\quad#2
	\end{flushright}
}

\newcommand{\aswtag}[1]{
	\quad\blacktriangleleft\quad#1
}

\title{\vspace{100pt}\Huge{HW3}}
\author{
	2025-1 고체역학(박성훈 교수님)\\[10pt]
	Problem 2.65, 2.101, 2.103, 2.105, 3.3, 3.11, 3.55\\[100pt]
	오류 제보\quad eusnoohong03@soongsil.ac.kr\\
	}
\date{\today}

\begin{document}
	
\renewcommand*{\titlepagestyle}{empty}
\maketitle

\vspace{60pt}

\begin{center}
	\includegraphics[width=0.45\textwidth]{SSU symbol KR-EN.jpg}
\end{center}

\newpage\setcounter{page}{1}

\setlength{\parindent}{0pt}

\prob{Problem 2.65}{In a standard tensile test, an aluminum rod of 20-mm diameter is subjected to a tension force of $P = 30\kn$. Knowing that $\nu = 0.35$ and $E = 70\gpa$, determine ($a$) the elongation of the rod in a 150-mm gage length, ($b$) the change in diameter of the rod.}

\begin{align*}
	&A = \frac{\pi}{4}d^2 = \frac{\pi}{4}(20\mm)^2 = 100\pi\mm^2\\
	&\delta = \frac{PL}{EA} = \frac{(30\times10^3)(0.15)}{(70\times10^9)(100\pi\times10^{-6})}\meter =  0.205\mm\aswtag{(a)}\\
	&\Delta d = \varepsilon_xd = -\nu\varepsilon_yd = -\frac{\nu Pd}{EA} = -\frac{(0.35)(30\times10^3)(0.02)}{(70\times10^9)(100\pi\times10^{-6})}\meter = -9.55\mum\aswtag{(b)}
\end{align*}

\vspace{10pt}

\probpic{Problem 2.101}{img/p02-101.png}{35mm}{The 30-mm-square bar $AB$ has a length $L = 2.2\meter$; it is made of a mild steel that is assumed to be elastoplastic with $E = 200\gpa$ and $\sigma_Y = 345\mpa$. A force $P$ is applied to the bar until end $A$ has moved down by an amount $\delta_m$. Determine the maximum value of the force $P$ and the permanent set of the bar after, the force has been removed, knowing that ($a$) $\delta_m = 4.5\mm$  ($b$) $\delta_m = 8\mm$.}

\begin{align*}
	&\delta_Y = \frac{\sigma_YL}{E} = \frac{(345\times10^6)(2.2)}{200\times10^9}\meter = 3.795\mm
\end{align*}
In part ($a$),
\begin{align*}
	&\delta > \delta_Y \quad\Rightarrow\quad (\text{elastic situation})\quad P_m = P_Y = \sigma_YA = (345\mpa)(30\mm)^2 = 310.5\kn\\
	&\delta_p = \delta_m - \delta_Y = 4.5\mm - 3.795\mm = 0.705\mm
\end{align*}
\asw{$P_m = 310.5\kn\;;\;\delta_p = 0.705\mm$}{($a$)}
In part ($b$),
\begin{align*}
	&\delta > \delta_Y \quad\Rightarrow\quad (\text{elastic situation})\quad P_m = P_Y = \sigma_YA = (345\mpa)(30\mm)^2 = 310.5\kn\\
	&\delta_p = \delta_m - \delta_Y = 8\mm - 3.795\mm = 4.205\mm
\end{align*}
\asw{$P_m = 310.5\kn\;;\;\delta_p = 4.205\mm$}{($b$)}

\newpage

\probpic{Problem 2.103}{img/p02-103.png}{60mm}{Rod $AB$ is made of a mild steel that is assumed to be elastoplastic with $E = 200\gpa$ and $\sigma_Y = 345\mpa$. After the rod has been attached to the rigid lever $CD$, it is found that end $C$ is 6 mm too high. A vertical force $Q$ is then applied at $C$ until this point has moved to position $C'$. Determine the required magnitude of $Q$ and the deflection $\delta_1$ if the lever is to \textit{snap} back to a horizontal positic sition after $Q$ is removed.}

\begin{align*}
	&\delta_p = \overline{BH} = \frac{0.7\meter}{1.1\meter}(6\mm) = 3.81818\mm,\quad P = \frac{0.7\meter}{1.1\meter}Q = \frac{7}{11}Q\\
	&P_m = P_Y = \sigma_YA = (345\mpa)\left\{\frac{\pi}{4}(9\mm)^2\right\} = 21.9480\kn\\
	&Q_m = \frac{7}{11}P_m = 13.97\kn\\
	&L = \overline{AB} = 1.25\meter - 3.81818\times10^{-3}\meter = 1.246182\meter\\
	&\delta_Y = \frac{\sigma_YL}{E} = \frac{(345\times10^6)(1.246182)}{200\times10^9}\meter = 2.14966\mm\\
	&\delta_1 = \frac{11}{7}\cdot\overline{HB'} = \frac{11}{7}(\delta_m - \delta_p) = \frac{11}{7}\delta_Y = 3.38\mm
\end{align*}
\asw{$Q_m = 13.97\kn\;;\;\delta_1 = 3.38\mm$}{}

\vspace{10pt}

\probpic{Problem 3.3}{img/p03-003.png}{60mm}{A 1.75-$\knm$ torque is applied to the solid cylinder shown. Determine ($a$) the maximum shearing stress, ($b$) the percent of the torque carried by the inner 25-mm-diameter core.}

\begin{align*}
	&J = \frac{\pi}{2}\cdot 0.025^4\meter^4 = 6.13592\times10^{-7}\meter^4\\
	&\tau_m = \frac{Tc}{J} = \frac{(1750)(0.025)}{6.13592\times10^{-7}}\nm = 71.3\mpa\aswtag{(a)}\\
	&J_\text{in} = \frac{\pi}{2}\cdot0.0125^4\meter^4 = \frac{1}{16}J\\
	&T = \frac{G\phi J}{L},\quad T_\text{in} = \frac{G\phi J_\text{in}}{L} = \frac{G\phi J}{16L} = \frac{T}{16},\quad \frac{T_\text{in}}{T} = \frac{1}{16} = 6.25\%\aswtag{(b)}
\end{align*}

\newpage

\probpic{Problem 3.11}{img/p03-011.png}{45mm}{The torques shown are exerted on pulleys $A$ and $B$. Knowing that both shafts are solid, determine the maximum shearing stress in
($a$) shaft $AB$, ($b$) shaft $BC$.}

\begin{align*}
	&T_{AB} = T_A = 300\nm,\quad T_{BC} = T_A + T_B = 700\nm\\
	&J_{AB} = \frac{\pi}{2}\cdot0.015^4\meter^4 = 7.95216\times10^{-8}\meter^4\\
	&J_{BC} = \frac{\pi}{2}\cdot0.023^4\meter^4 = 4.39573\times10^{-7}\meter^4\\
	&\tau_{m,AB} = \frac{T_{AB}c_{AB}}{J_{AB}} = \frac{(300)(0.015)}{7.95216\times10^{-8}}\pa = 56.6\mpa\aswtag{(a)}\\
	&\tau_{m,BC} = \frac{T_{BC}c_{BC}}{J_{BC}} = \frac{(700)(0.023)}{4.39573\times10^{-7}}\pa = 36.6\mpa\aswtag{(b)}
\end{align*}

\newpage

\probpic{Problem 3.55}{img/p03-055.png}{50mm}{Two solid steel shafts ($G = 77.2\gpa$) are connected to a coupling disk $B$ and to fixed supports at $A$ and $C$. For the loading shown, determine ($a$) the reaction at each support, ($b$) the maximum shearing stress in shaft $AB$, ($c$) the maximum shearing stress in shaft $BC$.}

\begin{align*}
	&J_{AB} = \frac{\pi}{2}\cdot 0.025^4\meter^4 = 6.13592\times10^{-7}\meter^4\\
	&J_{BC} = \frac{\pi}{2}\cdot 0.019^4\meter^4 = 2.04708\times10^{-7}\meter^4\\
	&T = T_{AB} + T_{BC},\quad T_{AB} = \frac{G\phi J_{AB}}{L_{AB}},\quad T_{BC} = \frac{G\phi J_{BC}}{L_{BC}}\quad\Rightarrow\quad T_{AB} : T_{BC} = \frac{J_{AB}}{L_{AB}} : \frac{J_{BC}}{L_{BC}}\\
	&\left.\begin{array}{l}
		T_{AB} = T\cdot\cfrac{\frac{J_{AB}}{L_{AB}}}{\frac{J_{AB}}{L_{AB}} + \frac{J_{BC}}{L_{BC}}} = \cfrac{TJ_{AB}}{J_{AB} + \frac{L_{AB}}{L_{BC}}\cdot J_{BC}} = 1105\nm\\[20pt]
		T_{BC} = T - T_{AB} = 1400\nm - 1105.062\nm = 295\nm\\
		\phantom{.}
	\end{array}\right\}\aswtag{(a)}\\
	&\tau_{m,AB} = \frac{T_{AB}c_{AB}}{J_{AB}} = 45.0\mpa\aswtag{(b)}\\
	&\tau_{m,BC} = \frac{T_{BC}c_{BC}}{J_{BC}} = 27.4\mpa\aswtag{(c)}
\end{align*}

\end{document}
\documentclass[a4paper]{scrartcl}
\usepackage[english]{babel}
\usepackage[top=2cm,bottom=3cm,left=2.5cm,right=2.5cm]{geometry}
\usepackage[colorlinks=true, allcolors=black]{hyperref}
\usepackage{wrapfig} %문단 내 이미지 삽입
\usepackage[abs]{overpic} %이미지 위 텍스트 삽입
\usepackage{graphicx} %색상
\usepackage{mdframed} %글상자
\usepackage[many]{tcolorbox}
	\newcommand{\asw}[2]{
		\begin{flushright}
			{\color{black}#1 \quad #2 \quad $\blacktriangleleft$}
		\end{flushright}
	}
\usepackage[yyyymmdd]{datetime}
	\renewcommand{\dateseparator}{-}
\usepackage{titlesec} %섹션 이름 변경
	\titlespacing*{\section}{0mm}{0mm}{0mm}
	\titleformat{\section}{\bfseries\Large}{\color{white}Chapter\hspace{1ex}\thesection}{3ex}{}[]
	\titlespacing*{\subsection}{0mm}{0mm}{0mm}
	\titleformat{\subsection}{\bfseries}{\color{white}\thesubsection}{3ex}{}[]

\newlength{\expwidth}
\newlength{\excwidth}
\newlength{\excwidtH}
\newlength{\secwidth}
\newlength{\sbswidth}

\newcommand{\secbox}[1]
{
	\settowidth{\secwidth}{\textbf{\Large{Chapter\hspace{1ex}\thesection}}}
	\begin{tcolorbox}[
		boxsep = 0pt,
		left = (-\secwidth -3mm),
		right = 2mm,
		boxrule = 0mm,
		leftrule = (\secwidth + 6mm),
		valign = center,
		colback = blue!70!black!15,
		colupper = blue!70!black!70,
		colframe = blue!70!black!70,
		arc = 0mm
		]{#1}
	\end{tcolorbox}
}
\newcommand{\sbsbox}[1]
{
	\settowidth{\sbswidth}{\textbf{\Large{\thesubsection}}}
	\begin{tcolorbox}[
		boxsep = 0pt,
		left = -\sbswidth,
		right = 2mm,
		boxrule = 0mm,
		leftrule = (\sbswidth + 2mm),
		valign = center,
		colback = yellow!50!red!12,
		colupper = black!80,
		colframe = red!70!black!70,
		arc = 0mm
		]{#1}
	\end{tcolorbox}
}
\newcommand{\expbox}[1]
{	
	\settowidth{\expwidth}{\textbf{문제 #1}}
	\begin{tcolorbox}[
		boxsep = 0pt,
		left = (-\expwidth-2mm),
		right = 2mm,
		top = 1mm,
		bottom = 0mm,
		boxrule = 0mm,
		bottomrule = 0.5mm,
		leftrule = (\expwidth + 4mm),
		valign = center,
		colback = white,
		colupper = white,
		colframe = red!70!black!65,
		arc = 0mm
		]{\textbf{문제 #1}}
	\end{tcolorbox}
}
\newcommand{\exercises}[1]
{	
	\settowidth{\excwidtH}{\textbf{#1}}
	\settowidth{\excwidth}{\textbf{Exercises}}
	\hspace{\excwidtH}
	\hspace{1mm}
	\begin{tcolorbox}[
		boxsep = 0pt,
		left = (-\excwidth-2mm),
		right = 2mm,
		height = 7mm,
		width = (\textwidth - \excwidtH - 5mm),
		boxrule = 0mm,
		bottomrule = 0.5mm,
		leftrule = (\excwidth + 4mm),
		valign = center,
		colback = yellow!50!red!12,
		colupper = white,
		colframe = red!70!black!65,
		arc = 0mm,
		enhanced,
		title = \textbf{#1},
		attach boxed title to top left = {
			xshift = (-\excwidtH - 5mm),
			yshift = -7mm
			},
		boxed title style = {
			size = small,
			boxrule = 0mm,
			colback = blue!70!black!70,
			height = 7mm,
			valign = center,
			arc = 0mm
			}
		]{\textbf{Exercises}}
	\end{tcolorbox}
}
\newcommand{\solution}{
	\begin{tabular}{m{8mm}m{152mm}}
		\vspace{1mm}{\color{red!70!black!65}\textbf{풀이}}
		&
		{\color{red!70!black!65}\rule{146mm}{0.5mm}}
	\end{tabular}
}
\newcommand{\exc}[1]{\noindent\hspace{2ex}\textbf{#1.}\quad}

\usepackage{amsmath, amsfonts, amssymb, bm} %수식
	\DeclareMathOperator{\arccsc}{arccsc}
	\DeclareMathOperator{\arcsec}{arcsec}
	\DeclareMathOperator{\arccot}{arccot}
	\DeclareMathOperator{\csch}{csch}
	\DeclareMathOperator{\sech}{sech}
	\DeclareMathOperator{\arcsinh}{arcsinh}
	\DeclareMathOperator{\arccosh}{arccosh}
	\DeclareMathOperator{\arctanh}{arctanh}
	\DeclareMathOperator{\arccsch}{arccsch}
	\DeclareMathOperator{\arcsech}{arcsech}
	\DeclareMathOperator{\arccoth}{arccoth}
	
	\DeclareMathOperator{\meter}{m}
	\DeclareMathOperator{\cm}{cm}
	\DeclareMathOperator{\mm}{mm}
	\DeclareMathOperator{\mum}{\mu m}
	\DeclareMathOperator{\newton}{N}
	\DeclareMathOperator{\kn}{kN}
	\DeclareMathOperator{\kgf}{kgf}
	\DeclareMathOperator{\pa}{Pa}
	\DeclareMathOperator{\kpa}{kPa}
	\DeclareMathOperator{\mpa}{MPa}
	\DeclareMathOperator{\gpa}{GPa}
	\DeclareMathOperator{\knpm}{kN/m}
	\DeclareMathOperator{\kph}{km/h}
	\DeclareMathOperator{\mps}{m/s}
	\DeclareMathOperator{\tkph}{kph}
	\DeclareMathOperator{\tmps}{mps}
	\DeclareMathOperator{\mpss}{m/s^2}
	\DeclareMathOperator{\dgr}{\!^\circ}
	\DeclareMathOperator{\cel}{\!^\circ C}
	\DeclareMathOperator{\kg}{kg}
	\DeclareMathOperator{\kgpcm}{kg/m^3}
	\DeclareMathOperator{\nm}{N\cdot m}
	\DeclareMathOperator{\kw}{kW}
	\DeclareMathOperator{\kwh}{kWh}
	\DeclareMathOperator{\mmhg}{mmHg}
	\DeclareMathOperator{\snd}{s}
\usepackage{polynom} %나눗셈 필산
\usepackage{cancel} %수식 약분선
\usepackage[normalem]{ulem}%취소선
\usepackage{array} %표
\usepackage{kotex} %한글

\title{\vspace{100pt}{\Huge 해설}}
\author{
	{\Large 고급공학수학2(가)(이동령 교수님) 중간고사}\\[10pt]
	{\Large 시험실시 : 2025-10-29 13:30-14:45 (75분)}\\[10pt]
	{\Large 평균점수 : 24/36}\\[90pt]
	{\Large 오류 제보 : eunsoohong03@soongsil.ac.kr}\\
}
\date{\today}

\begin{document}

\renewcommand*{\titlepagestyle}{empty}

\maketitle
\setlength{\parindent}{3mm}

\vspace{60pt}

\begin{center}
	\includegraphics[width=0.45\textwidth]{SSU symbol KR-EN.jpg}
\end{center}

\newpage\setcounter{page}{1}

\noindent \textbf{1-(a).}\quad 다음 행렬의 행렬식을 계산하시오.\quad $\mathbf{A} = \left(\begin{array}{rrr}1&-1&2\\2&-4&6\\3&-2&0\end{array}\right)$
	\begin{align*}
		&|\mathbf{A}| = 3\left|\begin{array}{rr}-1&2\\-4&6\end{array}\right| +2\left|\begin{array}{rr}1&2\\2&6\end{array}\right| = 3\cdot2 + 2\cdot2 = 10
	\end{align*}

\noindent \textbf{1-(b).}\quad 딸림 행렬 $\text{adj}\,\mathbf{A}$를 구하시오.
	\begin{align*}
		&\text{adj}\,\mathbf{A} = \left(\begin{array}{rrr}C_{11}&C_{12}&C_{13}\\C_{21}&C_{22}&C_{23}\\C_{31}&C_{32}&C_{33}\end{array}\right)^T = \left(\begin{array}{rrr}12&18&8\\-4&-6&-1\\2&-2&-2\end{array}\right)^T = \left(\begin{array}{rrr}12&-4&2\\18&-6&-2\\8&-1&-2\end{array}\right)
	\end{align*}

\noindent \textbf{1-(c).}\quad 다음 연립방정식을 푸시오.\quad $\left\{\begin{array}{c}x_1 - x_2 + 2x_3 = 1\\2x_1 - 4x_2 + 6x_3 = -3\\3x_1 - 2x_2 = 7\end{array}\right.$
	\begin{align*}
		&\left(\begin{array}{rrr|r}1&-1&2&1\\2&-4&6&-3\\3&-2&0&7\end{array}\right)\begin{array}{c}-2R_1 + R_2\\-3R_1 + R_3\\\Longrightarrow \end{array} \left(\begin{array}{rrr|r}1&-1&2&1\\0&-2&2&-5\\0&1&-6&4\end{array}\right)\begin{array}{c}-\frac{1}{2}R_2\\-R_2 + R_3\\\Longrightarrow\end{array} \left(\begin{array}{rrr|r}1&-1&2&1\\0&1&-1&\frac{5}{2}\\[3pt]0&0&-5&\frac{3}{2}\end{array}\right)\\
		&\Rightarrow\quad \begin{array}{r}x_1 -x_2 +2x_3 = 1\\x_2 -x_3 = \frac{5}{2}\\[3pt]-5x_3 = \frac{3}{2}\end{array}\quad\Rightarrow\quad\left(\begin{array}{r}x_1\\x_2\\x_3\end{array}\right) = \left(\begin{array}{r}3.8\\2.2\\-0.3\end{array}\right)
	\end{align*}

\newpage

\noindent \textbf{2-(a).}\quad 다음 행렬의 고유값을 모두 구하시오.\quad $\mathbf{A} = \left(\begin{array}{rrr}1&-2&-2\\-2&1&-2\\-2&-2&1\end{array}\right)$
	\begin{align*}
		&|\mathbf{A} - \lambda\mathbf{I}| = \left|\begin{array}{ccc}1-\lambda & -2& -2\\-2&1-\lambda&-2\\-2&-2&1-\lambda\end{array}\right| = (1-\lambda)\{(1-\lambda)^2 - 4\} +2(2\lambda-2-4) -2(4 -2\lambda + 2)\\
		& = -(\lambda-1)(\lambda-3)(\lambda+1) + 8(\lambda-3) = -(\lambda-3)(\lambda^2 - 1 - 8) = -(\lambda-3)^2(\lambda+3) = 0\\
		&\lambda_1 = -3,\quad \lambda_2 = 3
	\end{align*}
	
\noindent \textbf{2-(b).}\quad $\mathbf{A}$의 고유벡터들을 이용해서 직교행렬을 만드시오.
	\begin{align*}
		&\text{for }\lambda_1 = -3,\\
		&(\mathbf{A} +3\mathbf{I}|\mathbf{0}) = \left(\begin{array}{rrr|r}4&-2&-2&0\\-2&4&-2&0\\-2&-2&4&0\end{array}\right)
		\begin{array}{c}\frac{1}{2}R_1 + R_2\\[3pt]\frac{1}{2}R_1 + R_3\\\Longrightarrow\end{array}
		\left(\begin{array}{rrr|r}4&-2&-2&0\\0&3&-3&0\\0&-3&3&0\end{array}\right)
		\begin{array}{c}R_2 + R_3\\\frac{1}{3}R_2,\;\frac{1}{2}R_1\\\Longrightarrow\end{array}\\
		&\left(\begin{array}{rrr|r}2&-1&-1&0\\0&1&-1&0\\0&0&0&0\end{array}\right)
		\quad\Rightarrow\quad\begin{array}{r}2k_1 - k_2 - k_3 = 0\\k_2 - k_3 = 0\end{array}
		\quad\Rightarrow\quad k_1 = k_2 = k_3
		\quad\Rightarrow\quad \mathbf{K}_1 = \left(\begin{array}{r}1\\1\\1\end{array}\right)\\[10pt]
		&\text{for }\lambda_2 = 3,\\
		&(\mathbf{A} -3\mathbf{I}|\mathbf{0}) =
		\left(\begin{array}{rrr|r}-2&-2&-2&0\\-2&-2&-2&0\\-2&-2&-2&0\end{array}\right)
		\begin{array}{c}-R_1 + R_2\\-R_1 + R_3\\-\frac{1}{2}R_1\\\Longrightarrow\end{array}
		\left(\begin{array}{rrr|r}1&1&1&0\\0&0&0&0\\0&0&0&0\end{array}\right)
		\quad\Rightarrow\;k_1 + k_2 + k_3 = 0\\
		&\Rightarrow\quad\begin{array}{l}k_1 = t\\k_2 = u\\k_3 = -t-u\end{array}
		\quad\Rightarrow\quad\mathbf{K} = \left(\begin{array}{r}t\\u\\-t-u\end{array}\right) = t\left(\begin{array}{r}1\\0\\-1\end{array}\right) + u\left(\begin{array}{r}0\\1\\-1\end{array}\right)\\
		&\Rightarrow\quad \mathbf{K}_2 = \left(\begin{array}{r}1\\0\\-1\end{array}\right),\quad \mathbf{K}_* = \left(\begin{array}{r}0\\1\\-1\end{array}\right)\quad\mathbf{K}_*\text{와 }\mathbf{K}_2\text{가 직교하지 않음.}\\
		&\mathbf{K}_3' = \mathbf{K}_* - \frac{\mathbf{K}^T_*\mathbf{K}_2}{|\mathbf{K}_*||\mathbf{K}_2|}\mathbf{K}_2 = \left(\begin{array}{r}0\\1\\-1\end{array}\right) - \frac{1}{\sqrt{2}\cdot\sqrt{2}}\left(\begin{array}{r}1\\0\\-1\end{array}\right) = \left(\begin{array}{r}-\frac{1}{2}\\[2pt]1\\-\frac{1}{2}\end{array}\right),\quad \mathbf{K}_3 = -2\mathbf{K}_3' = \left(\begin{array}{r}1\\-2\\1\end{array}\right)\\
		&\mathbf{P} = \left(\frac{\mathbf{K}_1}{|\mathbf{K}_1|}\quad\frac{\mathbf{K}_2}{|\mathbf{K}_2|}\quad\frac{\mathbf{K}_3}{|\mathbf{K}_3|}\right) = \left(\begin{array}{ccr}
			\frac{1}{\sqrt{3}}&\frac{1}{\sqrt{2}}&\frac{1}{\sqrt{6}}\\[5pt]
			\frac{1}{\sqrt{3}}&0&-\frac{2}{\sqrt{6}}\\[5pt]
			\frac{1}{\sqrt{3}}&-\frac{1}{\sqrt{2}}&\frac{1}{\sqrt{6}}
		\end{array}\right),\quad \mathbf{P}^{-1} = \mathbf{P}^T
	\end{align*}
	
\noindent \textbf{2-(c).}\quad 대각행렬 $\mathbf{D}$에 대해 $\mathbf{D} = \mathbf{P}^{-1}\mathbf{AP}$를 만족시키는 $\mathbf{P}$, $\mathbf{D}$, $\mathbf{P}^{-1}$을 쓰시오.
	\begin{align*}
		&\mathbf{P} = \left(\begin{array}{ccr}
			\frac{1}{\sqrt{3}}&\frac{1}{\sqrt{2}}&\frac{1}{\sqrt{6}}\\[5pt]
			\frac{1}{\sqrt{3}}&0&-\frac{2}{\sqrt{6}}\\[5pt]
			\frac{1}{\sqrt{3}}&-\frac{1}{\sqrt{2}}&\frac{1}{\sqrt{6}}
		\end{array}\right),\quad \mathbf{D} = \left(\begin{array}{rrr}
		-3&0&0\\0&3&0\\0&0&3\end{array}\right)
		,\quad \mathbf{P}^{-1} = \left(\begin{array}{ccr}
			\frac{1}{\sqrt{3}}&\frac{1}{\sqrt{3}}&\frac{1}{\sqrt{3}}\\[5pt]
			\frac{1}{\sqrt{2}}&0&-\frac{1}{\sqrt{2}}\\[5pt]
			\frac{1}{\sqrt{6}}&-\frac{2}{\sqrt{6}}&\frac{1}{\sqrt{6}}
		\end{array}\right)
	\end{align*}
	
\newpage

\noindent \textbf{3.}\quad $\mathbf{A} = \left(\begin{array}{rr} 2&1\\0&1 \end{array}\right)$에 대해 $\mathbf{A}^{20}$을 구하시오.\\[10pt]
	풀이 1 | 대각화를 이용하여
	\begin{align*}
		&|\mathbf{A} - \lambda\mathbf{I}| = \left|\begin{array}{cc}2-\lambda&1\\0&1-\lambda\end{array}\right| = (\lambda-2)(\lambda-1) = 0\quad\Rightarrow\quad \begin{array}{r}\lambda_1 = 1\\\lambda_2=2\end{array}\quad\Rightarrow\quad \mathbf{D} = \left(\begin{array}{rr}1&0\\0&2\end{array}\right)\\
		&(\mathbf{A} - \mathbf{I}|\mathbf{0}) = \left(\begin{array}{rr|r}1&1&0\\0&0&0\end{array}\right)\quad\Rightarrow\quad k_1 + k_2 = 0\quad\Rightarrow\quad \mathbf{K}_1 = \left(\begin{array}{r}1\\-1\end{array}\right)\\
		&(\mathbf{A} -2\mathbf{I}|\mathbf{0}) = \left(\begin{array}{rr|r}0&1&0\\0&-1&0\end{array}\right)\quad\Rightarrow\quad k_2 = 0\quad\Rightarrow\quad \mathbf{K}_2 = \left(\begin{array}{r}1\\0\end{array}\right)\\
		&\mathbf{P} = \left(\begin{array}{rr}1&1\\-1&0\end{array}\right),\quad \mathbf{P}^{-1} = \left(\begin{array}{rr}0&-1\\1&1\end{array}\right)\\
		&\mathbf{A} = \mathbf{PDP}^{-1}\\
		&\mathbf{A}^{20} = \mathbf{PD}^{20}\mathbf{P}^{-1} = \left(\begin{array}{rr}1&1\\-1&0\end{array}\right)\left(\begin{array}{cc}1&0\\0&2^{20}\end{array}\right)\left(\begin{array}{rr}0&-1\\1&1\end{array}\right) = \left(\begin{array}{rc}1&2^{20}\\-1&0\end{array}\right)\left(\begin{array}{rr}0&-1\\1&1\end{array}\right)\\
		& = \left(\begin{array}{cc}2^{20}&2^{20}-1\\0&1\end{array}\right)
	\end{align*}
	풀이 2 | Cayley-Hamilton 정리를 이용하여
	\begin{align*}
		&|\mathbf{A} - \lambda\mathbf{I}| = \left|\begin{array}{cc}2-\lambda&1\\0&1-\lambda\end{array}\right| = \lambda^2 - 3\lambda + 2 = (\lambda-2)(\lambda-1) = 0\quad\Rightarrow\quad\begin{array}{r}\lambda_1 = 1\\\lambda_2=2\end{array}\\
		&\lambda^m = c_1\lambda + c_0\lambda\\
		&1^{20} = c_1+c_0\\
		&2^{20} = 2c_1+c_0\\
		&c_1 = 2^{20} - 1,\quad c_0 = -2^{20} + 2\\
		&\mathbf{A}^{20} = c_1\mathbf{A} +c_0\mathbf{I} = (2^{20}-1)\mathbf{A} -(2^{20}-2)\mathbf{I}\\
		&= \left(\begin{array}{cc}2^{21} - 2& 2^{20} - 1\\0&2^{20} - 1\end{array}\right) - \left(\begin{array}{cc}2^{20} - 2& 0\\0&2^{20} - 2\end{array}\right) = \left(\begin{array}{cc}2^{20}& 2^{20} - 1\\0&1\end{array}\right)
	\end{align*}
	
\newpage
	
\noindent \textbf{4.}\quad 다음 행렬을 단위위삼각행렬을 포함하도록 LU 인수분해하라.\quad $\mathbf{A} = \left(\begin{array}{rrr} -2&0&3\\6&1&-4\\-4&4&30 \end{array}\right)$
	\begin{align*}
		&\left(\begin{array}{rrr} -2&0&3\\6&1&-4\\-4&4&30 \end{array}\right)
		\begin{array}{c}3R_1 + R_2\\-2R_1 + R_3\\\Longrightarrow\end{array}
		\left(\begin{array}{rrr} -2&0&3\\0&1&5\\0&4&24 \end{array}\right)
		\begin{array}{c}-4R_2 + R_3\\\Longrightarrow\end{array}
		\left(\begin{array}{rrr} -2&0&3\\0&1&5\\0&0&4 \end{array}\right) = \mathbf{U}'\\
		&\mathbf{L}' = \left(\begin{array}{rrr} 1&0&0\\-3&1&0\\2&4&1 \end{array}\right)\\
		&\mathbf{A} = \mathbf{L}'\mathbf{U}' = \mathbf{L}'\mathbf{DD}^{-1}\mathbf{U}' = \left(\begin{array}{rrr} 1&0&0\\-3&1&0\\2&4&1 \end{array}\right)\left(\begin{array}{rrr} -2&0&0\\0&1&0\\0&0&4 \end{array}\right)\left(\begin{array}{rrr} -\frac{1}{2}&0&0\\0&1&0\\0&0&\frac{1}{4} \end{array}\right)\left(\begin{array}{rrr} -2&0&3\\0&1&5\\0&0&4 \end{array}\right)\\
		&\mathbf{L}'\mathbf{D} = \left(\begin{array}{rrr} -2&0&0\\6&1&0\\-4&4&4 \end{array}\right) = \mathbf{L}\\
		&\mathbf{D}^{-1}\mathbf{U}' = \left(\begin{array}{rrr} 1&0&-\frac{3}{2}\\[3pt]0&1&5\\0&0&1 \end{array}\right) = \mathbf{U}\\
		&\mathbf{A} = \left(\begin{array}{rrr} -2&0&0\\6&1&0\\-4&4&4 \end{array}\right)\left(\begin{array}{rrr} 1&0&-\frac{3}{2}\\[3pt]0&1&5\\0&0&1 \end{array}\right)
	\end{align*}
	
\newpage

\noindent \textbf{5.}\quad $t = \pi/2$에서 $\mathbf{r}(t)= 2\cos t\mathbf{i} + 2\sin t\mathbf{j} + 3t\mathbf{k}$의 단위접선벡터, 단위법선벡터, 단위종법선벡터, 접촉평면, 법평면, 전직평면의 방정식을 구하시오.
	\begin{align*}
		&\mathbf{T}(t) = \frac{\mathbf{r}'(t)}{|\mathbf{r}'(t)|} = \frac{-2\sin t\mathbf{i} + 2\cos t \mathbf{j} + 3\mathbf{k}}{\sqrt{13}},\quad \mathbf{T}\left(\frac{\pi}{2}\right) = \frac{-2\mathbf{i} +3\mathbf{k}}{\sqrt{13}}\quad\blacktriangleleft\quad\text{단위접선벡터}\\
		&\mathbf{N}(t) = \frac{\mathbf{T}'(t)}{|\mathbf{T}'(t)|} = \frac{\frac{1}{\sqrt{13}}(-2\cos t\mathbf{i} -2\sin t\mathbf{j})}{\frac{1}{\sqrt{13}}\cdot2} = -\cos t\mathbf{i} -\sin t\mathbf{j},\quad \mathbf{N}\left(\frac{\pi}{2}\right) = -\mathbf{j}\quad\blacktriangleleft\quad\text{단위법선벡터}\\
		&\mathbf{B} = \mathbf{T}\times\mathbf{N} = \frac{1}{\sqrt{13}}\left|\begin{array}{rrr}\mathbf{i}&\mathbf{j}&\mathbf{k}\\-2&0&3\\0&-1&0\end{array}\right| = \frac{3\mathbf{i} +2\mathbf{k}}{\sqrt{13}}\quad\blacktriangleleft\quad\text{단위종법선벡터}\\
		&\mathbf{r}\left(\frac{\pi}{2}\right) = 2\mathbf{j} + \frac{3}{2}\pi\mathbf{k} = \;<0,2,\frac{3}{2}\pi>\\
		&\left(<x,y,z> - <0,2,\frac{3}{2}\pi>\right)\cdot\mathbf{B} = 0\quad\Rightarrow\quad
		\frac{1}{\sqrt{13}}\left\{3x +2\left(z - \frac{3}{2}\pi\right)\right\} = 0\quad\Rightarrow\quad
		3x +2z = 3\pi\\
		&\hspace{128mm}\blacktriangle\quad\text{접촉평면}\\
		&\left(<x,y,z> - <0,2,\frac{3}{2}\pi>\right)\cdot\mathbf{T} = 0\quad\Rightarrow\quad\frac{1}{\sqrt{13}}\left\{-2x +3\left(z - \frac{3}{2}\pi\right)\right\} = 0\quad\Rightarrow\quad -4x + 6z = 9\pi\\
		&\hspace{132mm}\blacktriangle\quad\text{법평면}\\
		&\left(<x,y,z> - <0,2,\frac{3}{2}\pi>\right)\cdot\mathbf{N} = 0\quad\Rightarrow\quad-(y-2)= 0\quad\Rightarrow\quad y=2\quad\blacktriangleleft\quad\text{전직평면}
	\end{align*}

\newpage

\noindent \textbf{6.}\quad 점 (0,2,1)에서 함수 $f(x,y,z) = 2xy + z^2e^y$의 $\mathbf{v} = 1\mathbf{i} + 2\mathbf{j} -2\mathbf{k}$방향으로의 방향도함수를 구하시오.
	\begin{align*}
		&\mathbf{u} = \frac{\mathbf{v}}{|\mathbf{v}|} = \frac{1}{3}(\mathbf{i} + 2\mathbf{j} -2\mathbf{k})\\
		&\nabla f(x,y,z) = 2y\mathbf{i} + (2x + z^2e^y)\mathbf{j} + 2ze^y\mathbf{k}\\
		&\nabla f(0,2,1) = 4\mathbf{i} + e^2\mathbf{j} + 2e^2\mathbf{k}\\
		&D_{\mathbf{u}}f(0,2,1) = \mathbf{u}\cdot\nabla f(0,2,1) = \frac{1}{3}(4 + 2e^2 -4e^2) = \frac{2}{3}(2 - e^2)
	\end{align*}
	




\end{document}